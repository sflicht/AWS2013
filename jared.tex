\documentclass{article}
\usepackage{sagetex}
\usepackage{amsmath}
\usepackage{amssymb}

\begin{document}


\section{Basic setup}
We wish to compute in the ring $\mathcal O_L = \mathbf{F}_q[[t^{1/q^\infty}]]$,
the $t$-adic completion of $\mathbf{F}_q[t^{1/q^\infty}]$. For concreteness we set $q=3$,
and  begin by constructing $\mathbf{F}_q$ and $\mathbf{F_q}^\times$ in Sage:
\begin{sageblock}
q = 2
Fq.<a> = GF(q)
Fqx = Fq.list()[:]; Fqx.remove(0)
\end{sageblock}
For computational purposes, we bound the power of $q$ that can occur in the denominators
of the exponents of powers of $t$:
\begin{sageblock}
bound = 10
\end{sageblock}
Now we construct the ring $R=\mathbf{F}_q[t^{1/q^{bound}}]$:
\begin{sageblock}
vars = ["t"] + ["t_"+str(i) for i in range(1,bound+1)]
R0 = PolynomialRing(Fq,vars)
I = R0.ideal([R0(vars[i])^q-R0(vars[i-1]) for i in range(1,bound+1)])
R = R0.quo(I)
\end{sageblock}
We will abbreviate $t^{q^{1/n}}$ by $t_n$:
\begin{sageblock}
t = R.gens()[0]
t_1,t_2,t_3,t_4 = var("t_1,t_2,t_3,t_4")
[t_1,t_2,t_3,t_4] = R.gens()[1:5] 
\end{sageblock}
For example, we can compute in Sage that, in $R$, we have $t_4^{q^4}-t = \sage{t_4^(q^4)-t}$,
and likewise test that $tt_1 = t_1^{q+1}: \sage{t*t_1==t_1^(q+1)}$.

Now we can't actually work in the completion $\mathcal O_L$ of $R$ in Sage,
but we can at least construct the normalized $t$-adic valuation.
For example, we want $v(t_2^5+t_4^{q^{10}}) = 5/q^2=\sage{5/q^2}$.
Here's a slick way to code this:
\begin{sageblock}
def v(r):
    S.<x> = PolynomialRing(Fq)
    phi = R0.hom([x]+[0 for i in range(bound)],S)
    s0 = (r^(q^bound)).lift() # lift r to a polynomial
    s = phi(s0)
    return s.valuation() / (q^bound)
\end{sageblock}    
Unfortunately, this won't work when bound is reasonable large, because taking large powers is a no-no in Sage.

So here is a hacky solution:
\begin{sageblock}
print "Defining v(r)"
def v(r):
    if r == 0:
        return Infinity
    r1 = r/r.lc()
    mvs = []
    d = zip(R.variable_names(),
                 ["("+str(q^(-i))+")" for i in range(len(R.variable_names()))])
    #       probably should build this dict once and store it, 
    #       rather than creating it every time v(r) is called
    d.append(("*","+"))
    d.append(("^","*"))
    while r1 != 0:
        r1 = r1/r1.lc()
        #print r1.lift()
        m = r1.lm() # get the normalized leading monomial
        ms = str(m) # turn this into a string
        for dd in d:
            ms = ms.replace(dd[0],dd[1])
        mv = QQ(sage.calculus.calculus.symbolic_expression_from_string(ms)) 
       # mv is the valuation of the leading monomial of r1        
        mvs.append(mv)
        r1 = r1 - m
    return min(mvs)
\end{sageblock}


Let 
\begin{sageblock}
r = var("r") 
r = t_2^5 + t_4^(q^10)
\end{sageblock}
Then $v(r) = \sage{v(r)}$.

\section{A hopefully convergent sequence}
Let $K = \mathbf{F}_q((\pi))$ and let $\mathcal F$ be a Lubin--Tate formal $\mathcal O_K$-module.
Setting $K_n = K(\mathcal F[\pi^n]), K_\infty=\bigcup K_n$, we have $L = \mathrm{Frac}(\mathcal O_L) = \hat{K}_\infty$, and the galois group $H=\mathrm{Gal}(K_\infty/K)$ acts on $L$.

We define $A_n = \mathbf{F}_q^\times \times \mathbf{F_q}^{n-1}$:
\begin{sageblock}
def A(n):
    return map(flatten, CartesianProduct(Fqx,map(list,VectorSpace(Fq,n-1).list())))
\end{sageblock}
Jared conjectures that
\[\pi = \lim_{n\to\infty} \prod_{(a_0,\ldots, a_{n-1})\in A} (\sum a_i t^{q^i})^{q^{-n}}\]
should converge to an element $\pi\in\mathcal{O}_L$ which is fixed by the action of $H$.
Write $\pi_n$ for the expression in the limit.
\begin{sageblock}
def pi(n):
    t_n = R.gens()[n] 
    return prod([sum([a[i]*t_n^(q^i) for i in range(n)]) for a in A(n)])
\end{sageblock}

\begin{sagesilent}
print pi(1)
print pi(2)
print pi(3)
def prettyPrint(r):
    d = zip(map(str,R.gens()),vars)
    s = str(r)
    for dd in d:
        s = s.replace(dd[0],dd[1])
    r.rename(s)
    return r
\end{sagesilent}
 
Examples:
\begin{sageblock}
pi_1 = pi(1)
print 1
pi_2 = pi(2)
print 2
pi_3 = pi(3)
print 3
pi_4 = pi(4)
print 4
pi_5 = pi(5)
#print 5
pi_6 = pi(6)
#print 6
#pi_7 = pi(7)
#print 7
#pi_8 = pi(8)
#print 8
#pi_9 = pi(9)
#print 9
\end{sageblock}
% It is necessary to lift the elements of R to elements of the polynomial
% ring R0 (defined above), in order to get Sage to TeXiFy them properly.
\[\pi_1 = \sage{pi_1.lift()}\]
\[\pi_2 = \sage{pi_2.lift()}\]
\[\pi_3 = \sage{pi_3.lift()}\]
%\[\pi_4 = \sage{pi_4.lift()}\]
%\[\pi_5 = \sage{pi_5.lift()}\]

Hmm, it is difficult to tell whether these are converging!
\[v(\pi_2-\pi_1) = \sage{v(pi_2-pi_1)}\]
\[v(\pi_3-\pi_2) = \sage{v(pi_3-pi_2)}\]
\[v(\pi_4-\pi_3) = \sage{v(pi_4-pi_3)}\]
\[v(\pi_5-\pi_4) = \sage{v(pi_5-pi_4)}\]
\[v(\pi_6-\pi_5) = \sage{v(pi_6-pi_5)}\]
%\[v(\pi_7-\pi_6) = \sage{v(pi_7-pi_6)}\]
%\[v(\pi_8-\pi_7) = \sage{v(pi(8)-pi(7))}\]
%\[v(\pi_9-\pi_8) = \sage{v(pi(9)-pi(8))}\]
%\[v(\pi_{10}-\pi_9) = \sage{v(pi(10)-pi(9))}\]

The data for a few small choices of $q$:
\[q=2: v(\pi_{n+1}-\pi_n) = \frac24+\frac n4,n=1,\ldots, 8\]
\[q=3: v(\pi_{n+1}-\pi_n) = \frac59+\frac{4n}{9},n=1,\ldots,5\]
\[q=4: v(\pi_{n+1}-\pi_n) = \frac{12}{16} +\frac{9n}{16},n=1,\ldots, 3\]
\[q=5: v(\pi_{n+1}-\pi_n) = \frac{20}{25}+\frac{16n}{25},n=1,\ldots,3\]


OK, this does look like a Cauchy sequence! Of course, that could be an artifact 
of working with just this small example. Can we prove it?
\end{document}


